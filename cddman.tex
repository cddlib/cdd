% The name of this file: cddman.tex
% written by by Komei Fukuda
% December 14, 1993
%
\documentstyle[11pt]{article}
\renewcommand{\baselinestretch}{1.0}
\setlength{\oddsidemargin}{8mm}
\setlength{\textwidth}{16cm}
\setlength{\topmargin}{0mm}
\setlength{\textheight}{23cm}
\setlength{\headsep}{0in}
\setlength{\headheight}{0pt}

\pagestyle{empty}

\begin{document}

.\vspace{20mm}

\begin{center}

% {\LARGE RO931123}


\vspace{20mm}

{\LARGE cdd User Manual}

\vspace{20mm}

{\Large Komei Fukuda\\
   EPFL, Lausanne, Switzerland\\
   fukuda@dma.epfl.ch\\
   and\\
   University of Tsukuba, Tokyo, Japan\\
  fukuda@gssm.otsuka.tsukuba.ac.jp
}

\vspace{20mm}
{\Large (cdd Version 0.27,  December 14, 1993)}

\end{center}

\newpage
\pagestyle{plain}
\setcounter{page}{1} 

\begin{center}

{\Large cdd User Manual}

\bigskip
{\large Komei Fukuda\\
   fukuda@dma.epfl.ch  or  fukuda@gssm.otsuka.tsukuba.ac.jp\\
   EPFL, Lausanne, Switzerland, and\\
   University of Tsukuba, Tokyo, Japan}

\bigskip
{\large (cdd Version 0.26,  November 30, 1993)}

\end{center}

\section{Introduction} \label{INTRODUCTION}

The program  cdd.c  is a C implementation of 
the Double Description Method~\cite{MRTT}
for generating all vertices (i.e. extremal points)
and extremal rays of a general d-dimensional 
convex polyhedron given by a system of linear inequalities:
\[
   P = \{ x  \in R^d:  A  x  \le  b \}
\]
where $A$ is an $m \times d$ real matrix and $b$ is a real
$m$ dimensional vector. 
 
This program has been produced
from the pascal code pdd.p (Version 0.2)
which was written by the author.   For the translation,
we applied the Pascal-to-C translator p2c
by Dave Gillespie to cdd.c, and we then modified
the C-code so that it runs without any p2c libraries.

One major advantage of this C-version over the pascal version is
that there is no restriction on the size  $m$ (and $d$), while in pdd.p
the size $m$ must be smaller than equal to the maximum element size
of SET TYPE variables in Pascal compiler used.

At the same time, cdd works just like pdd otherwise.
The program cdd reads input and writes output in 
{\em Polyhedra format\/} which was defined by David Avis and
the author.  The program called rs developed by David Avis is
a C-implementation of the reverse search algorithm~\cite{AF} 
for the same enumeration purpose, and it conforms to Polyhedra format as well.
Hopefully, this compatibility of the two programs
enables users to use both programs for the same input files
and to choose whichever is useful for their purposes.
From our experiences with relatively large problems,
the two methods are both useful and perhaps complementary
to each other.

One interesting feature of cdd is its capability
of handling the dual (reverse)  problem without any transformation
of data.  The dual problem is known to be the {\em hull problem\/} which
is to obtain a linear inequality representation
of a convex polyhedron given as the Minkowski sum of 
the convex hull of a finite set of points and the nonnegative
hull of a finite set of points in $R^d$: 
$P = conv(v_1,\ldots,v_n) +  nonneg(r_1,\ldots,r_s)$.
As we see in this manual, the computation can be done
in straightforward manner.

The program cdd seems quite good in handling degenerate
polyhedra.  Among the largest problems which could be
solved is a 21-dimensional hull problem given by 63 
extremal rays. This polyhedral cone, known as the {\em complete
cut cone on 7 points\/}, has exactly 38,780 facets and
some of facets contain many rays, see e.g. \cite{G}. 
It took a couple of days to compute the rays exactly on
a fast workstation.  The input file (ccc7.ine) of this polyhedra is
included in the distribution.

The size of an input file hardly indicates the degree of 
hardness of its vertex/ray enumeration.  While this program
can handle a highly degenerate problem (prodmT5.ine) with 
711 inequalities in  19 dimension quite easily with
the computation time 3-4 minutes on fast workstation, 
a 8-dimensional problem (mit729-9.ine) with 729 inequalities
can be extremely hard.  It took us about a month on
a reasonably fast Sun 4 workstation.  The latter problem arises
from the ground state analysis of a ternary alloy model, see \cite{CGAF}.
Both input files are included in the distribution.  

Although the program can be used for nondegenerate inputs,
it might not be very efficient.  For nondegenerate inputs, 
the use of the reverse search code rs is strongly
recommended.

One serious drawback of the current cdd is that the computation
is done with floating point arithmetic, and it does
perform infinite precision arithmetic. In contrast, the program rs
supports multiple precision arithmetic with any fixed
multiplicity.  We are now developing
a new version of cdd, which will hopefully perform the exact
computation for integer and rational inputs.

This program can be distributed freely under the GNU GENERAL PUBLIC LICENSE.
Please read the file COPYING carefully befor using.

I will not take any responsibility of any problems you might have
with this program.  But I will be glad to receive bug reports or suggestions
at the e-mail addresses above. 

\section{Polyhedra Input and Output Format} \label{FORMAT}
\bigskip
Polyhedra input format is quite simple:

\begin{tabular}{ccl}
\\ \hline
\multicolumn{3}{l} {various comments}\\
\multicolumn{3}{l} {\bf begin}\\
 $m$ & $d+1$ & {\bf numbertype}\\
 $b$ & $-A$ \\
\multicolumn{3}{l} {\bf end}\\
\multicolumn{3}{l} {various options} \\ \hline
\end{tabular}

\bigskip
\noindent
where numbertype can be one of integer, rational or real.
There is one restriction: the last $d$ rows must determine
a vertex of $P$.  The program cdd does not care whether
this condition is satisfied, as long as the polyhedron
contains at least one vertex.  But one can specify that 
the last $(d+1)$ rows be chosen as the initial set of
$(d+1)$ rows for the double description algorithm.
See {\bf initbasis\_at\_bottom} option in Section~\ref{OPTIONS}.

Polyhedra output format is quite similar to input format:

\begin{tabular}{ccl}
\\ \hline
\multicolumn{3}{l} {various comments produced by a program}\\
\multicolumn{3}{l} {\bf begin}\\
 $n+s$ & $d+1$ & {\bf numbertype}\\
 $1$ & $v_1$  & \\
 $\vdots$ & $\vdots$  & \\
 $1$ & $v_n$  & \\
 $0$ & $r_1$  & \\
 $\vdots$ & $\vdots$  & \\
 $0$ & $r_s$  & \\
\multicolumn{3}{l} {\bf end}\\  \hline
\end{tabular}

\bigskip
\noindent
where $v_1, \ldots, v_n$ are the vertices and
$r_1, \ldots, r_s$ are the extremal rays of the
polyhedron $P$.  Here we do not require that
the vertex list and the ray list are output
separately; they can appear mixed in arbitrary
order.


For example, let $P$ be the following unbounded 3-dimensional 
polyhedron given by
\[
   P = \{ x  \in R^3:
    1\le x_1 \le 2, \; 1 \le x_2 \le 2, \; 1 \le x_3\},
\]
which is a 3-cube without one ^^ ^^ lid". For finding all
vertices and extremal rays, the input file for cdd is

\begin{verbatim}
file name: ucube.ine
3 cube without one "lid"
begin
        5      4     integer
  2   -1   0   0
  2    0  -1   0
 -1    1   0   0
 -1    0   1   0
 -1    0   0   1
end
incidence
\end{verbatim}

The meaning of option ^^ ^^ incidence" will be explained
in Section~\ref{OPTIONS}.
After you run cdd with this input file, you will get
an output file, say ucube.ext, which looks like:

\begin{verbatim}
* cdd: Double Description Method C-Code:Version C0.27 (December 8, 1993)
* Copyright (C) 1993, Komei Fukuda, fukuda@dma.epfl.ch
*Input File:ucube.ine   (   5  x    4)
*HyperplaneOrder: LexMin
*AdjacencyTest: Combinatorial
*Degeneracy preknowledge for computation: None (possible degeneracy)
*Computation starts     at Mon Dec 13 14:46:18 1993
*            terminates at Mon Dec 13 14:46:18 1993
*Total processor time = 0 seconds
*                     = 0 hour 0 min 0 sec
*FINAL RESULT:
*Number of Vertices =       4,   Rays =       1
begin
        5      4    real
  0  0  0  1
  1  1  1  1
  1  2  1  1
  1  1  2  1
  1  2  2  1
end
\end{verbatim}

The output shows that the polyhedron has four vertices
$(1,1,1)$, $(2,1,1)$, $(1,2,1)$,  $(2,2,1)$ and
only one extremal ray $(0,0,1)$.  The comments contain
information on the name of input file, and the options
chosen to run the program which will be explained in
the next section.

\newpage
\section{Options}  \label{OPTIONS}

The following options are available for cdd.  These options are
set if they appear in input file after the ^^ ^^ end" command.
Independent options can be set simultaneously, but each option
must be written separately in one line, and  two options
should not be written in one line.

\begin{description}

\item[hull] option \\
When this option is chosen, the program cdd
will do the reverse operation.   That is, the input is assumed to
be a set of points and directions (rays).  When this option is set,
it is required that each data line must start with 
either ^^ ^^ 1" or ^^ ^^ 0", meaning points and rays, respectively.
More specifically, if the input file is of form

\begin{tabular}{ccl}
\\ \hline
\multicolumn{3}{l} {comments}\\
\multicolumn{3}{l} {\bf begin}\\
 $n+s$ & $d+1$ & {\bf numbertype}\\
 $1$ & $v_1$  & \\
 $\vdots$ & $\vdots$  & \\
 $1$ & $v_n$  & \\
 $0$ & $r_1$  & \\
 $\vdots$ & $\vdots$  & \\
 $0$ & $r_s$  & \\
\multicolumn{3}{l} {\bf end}\\
\multicolumn{3}{l} {\bf hull} \\ \hline
\end{tabular}

\noindent
Then the input is interpreted as the polyhedron in $R^d$:\\
$P = conv(v_1,\ldots,v_n) +  nonneg(r_1,\ldots,r_s)$ 
and the output will be a minimal system of linear inequalities
to represent $P$. 

\item[dynout\_off] option\\
When this option is chosen, the program will
not output vertices and rays to the CRT in real time.  
The default is dynout\_on.

\item[stdout\_off] option\\
 When this option is chosen, the program will not
output any progress report of computation (iteration number. etc).
The default is stdout\_on.

\item[logfile\_on] option\\
 When this option is chosen, the program will output
to a specified file (*.ddl) some information on the computation history.
This can be useful when the user does not know which hyperplane order
(mincutoff, maxcutoff, mixcutoff, lexmin, lexmax, minindex)
is efficient for computation.

\item[incidence, \#incidence] options\\
When the {\bf incidence} option is selected, the incidence relations between
the vertices/rays and the inequalities will be output
in a separate file (*.icd).  Here, a vertex is said to be 
{\em incident with\/} an inequality if the inequality is satisfied by equality.
An extremal ray $r$ is said to be {\em incident with\/} 
an inequality $a^T \; x \le b$ if  $a^T \; r = 0$.  For example,
since the incidence option was set for the example input file ucube.ine in
the previous section, the program outputs the following ucube.icd file:
\begin{verbatim}
*Input File:ucube.ine   (   5  x    4)
*Incidences of output(=vertices/rays) and input (=hyperplanes)
*   for each output, #incidence and the set of hyperplanes containing it
*   or its complement with its cardinality with minus sign
begin
       5    5    6
      -1    5
       3    3    4    5
       3    1    4    5
       3    2    3    5
       3    1    2    5
end
\end{verbatim}
After ^^ ^^ begin", there are three numbers $5 \quad 5 \quad 6$.
The first number $5$ is a number of output (vertices and rays).
The next number $5$ is $m$, the number of inequalities in the input file.
The last number $6$ is usually $m+1$, and $m$ if the hull option is chosen.
The number $m+1$ corresponds to the infinity constraint which is added
for vertex/ray enumeration.

The incidence data starts right after these three numbers.
At each line, the cardinality of incident inequalities and
the list of their indices are given.  There is an exception that, when
there are more incident inequalities than non-incident ones, then the program
outputs the list of non-incident inequalities with its
size with negative sign.  This is to save space of output.

For example, the second output line $3 \quad 3 \quad 4 \quad 5$ 
corresponds to the
second vertex of ucube.ext file in previous section, that is, 
the vertex $(1, 1, 1)$.  The first number $3$ is simply the number
of incident inequalities and the rest is the indices of
those inequalities, and so the 3rd, 4th and 5th inequalities are
satisfied by equality at this vertex.   The first output
$-1 \quad 5$ corresponds to the ray  $(0,0,1)$.  Since all inequalities
except the last (5th) inequality are incident with this ray,
the output is the complementary list with its cardinality (=1) with negative
sign.  Note that the original list is $1 \: 2 \: 3 \: 4 \: 6$, where
$6$ is the infinity plane.  One can ignore the infinity
plane for some purposes,  but for analyzing the combinatorial 
structure of polyhedra, it is
very important information.

The {\bf \#incidence} option can be used when you do not wish
to output the incidence file but to output only the cardinality of incidence
for each output, at the end of each output line.

\item[nondegenerate] option\\
When this option is set, the program assumes that the input system
is not degenerate, i.e., each vertex of the polyhedron satisfies
exactly $d$ inequalities of input with equality, 
and each ray satisfies exactly $d-1$ inequalities of input with equality.
It will run much faster with this option, but of course, 
if this option is set for degenerate inputs, it is 
quite possible that the output is incorrect.  
The default is this option being off.

\item[mincutoff, maxcutoff, mixcutoff, lexmin, lexmax, minindex] options\\
The double description is an incremental algorithm which
computes the vertices/rays of a polyhedron given by some $k$ of
original inequalities from the precomputed vertices/rays of a
polyhedron given by $k-1$ inequalities.  It is observed that
the efficiency of the algorithm depends strongly on how
one selects the ordering of inequalities, although a little
can be said theoretically.
These options are to select the ordering of inequalities to be
added at each iteration, and it is recommended to do small
experiment to select good ordering for a specific type of problems.
Unfortunately, a good ordering depends on the problem and there does not seem 
to be THE BEST ordering for every computation.  From our experiences,
lexmin, lexmax, mincutoff, maxcuoff work quite well in general.


The default is lexmin ordering which simply order inequalities
with respect to lexico-graphic ordering of rows of $A$.  The lexmax
is reverse of lexmin.  The mincutoff (maxcutoff) option selects an inequality which
cuts off the minimum (maximum) number of vertices/rays of the $k-1$th polyhedron. 
The mixcutoff option is the mixture of mincutoff and maxcutoff which selects
an inequality which cuts off the $k-1$th polytope as unbalanced as possible.
The maxcutoff option might be efficient if the input contains
many redundant inequalities (many interior points for hull computation).
Finally, the minindex option selects the hyperplanes from the top of
the input.

\item[initbasis\_at\_bottom] option\\
When this option is set, the program tries to select
the initial set of rows for the double description
method from the bottom of the input.  This means that
if the last (d+1) rows are independent, 
they will be chosen to initiate the algorithm.

This option is {\em not\/} default. The default
follows the same ordering as the ordering
of inequalities chosen.  This means that if {\bf lexmin\/}
is the ordering of inequalities, then the initial 
independent rows
will be chosen sequentially with lexico-min ordering.
There are exceptions when this rule
is not applicable, i.e. when one of mincutoff or maxcutoff
options is chosen. In such cases, {\bf lexmin\/}
ordering will be chosen.

\end{description}

\section{Source Files and Compilation}  \label{CAUTIONS}

\begin{itemize}

\item[(1)] [Files and Compilation] The source files for distribution are

\begin{tabular}{ll}
cdd.c  &  the cdd C source code \\
setoper.c  &  a set operation library code \\
setoper.h  &  the header file for setoper.c \\
cddman.tex &  the latex source file of this manual \\
cddHISTORY &  brief description of changes made at each updates \\
ccc5.ine, etc.  &  sample input files (stored in ^^ ^^ ine" subdirectory)\\
ccc5.ext, etc.  &  sample output files (stored in ^^ ^^ ext" subdirectory)\\
COPYING    &  GNU GENERAL PUBLIC LICENSE.\\
\end{tabular}

\noindent
To compile the code in standard unix environment, type
\begin{verbatim}
  cc -O cdd.c setoper.c
\end{verbatim}

to obtain an optimized executable file a.out. 
Since the program includes some standard ANSI library 
headers such as stdlib.h and time.h at compilation, 
the compiler must know the locations of the standard ANSI libraries.
Also, the files cdd.c, setoper.c and setoper.h are supposed to be
in the current directory.

\item[(2)] [Recompilation] The first two constants in the program cdd.c are to be 
changed by the user, and the program must be recompiled
each time any change is made.   These constants are simply
to specify the largest size of acceptable input data $(b, -A)$: 
\begin{verbatim}
#define mmax     201  /* USER'S CHOICE: max row size of A plus one */
#define nmax     24   /* USER'S CHOICE: max column size of A plus one */
\end{verbatim}
If this input data has $m$ rows and $d+1$ columns, then in the program,
mmax should be at least $m+1$ and $nmax$ should be at least
$d+1$.  It is strongly recommended to keep the size  nmax  as small as
possible because any temporary vertices and rays are stored with
arrays of this size.  Also, the set operation is performed frequently
by the program cdd on sets with size mmax, and thus the smaller the size mmax
is, the faster the program runs. 

Unlike the pascal version pdd, one can set the size mmax as large as one
wants.  It is no more restricted by the SET TYPE element sizes of 
usual Pascal compilers.

\item[(3)] [TURBO/THINK C Users] The program cdd.c is written in
ANSI C, and hence it should run on
personal computers without any changes if one uses a compliler supportin
ANSI standard. We have verified that it can be compiled by THINK Pascal without
any problems if ANSI library is added to the project together with
cdd.c and setoper.c.  We have not yet tried to compile it with TURBO C.

\item[(4)] [Rational Input Not Supported] The present cdd.c does not
read any RATIONAL numbers ($a/b$ type with $a$, $b$ integers).


\end{itemize}

\bigskip
\noindent
{\large {\bf Acknowledgements.}} 
I am  grateful for Prof. Th. M. Liebling of
Department of Mathematics, EPFL, Switzerland who
provided me with an ideal opportunity to visit the department
for the academic year 1993-1994.  Without his 
generous help and encouragement,
the present form of this software would not have existed.
Also I am indebted to Dr. Alain Prodon and Mr. Francois Margot
of EPFL who used earlier versions of the program extensively,
found several bugs, and  gave me very helpful suggestions.
I am thankful to Dr. Henry Crapo of INRIA, France for
his precious advise on the input and output style of Polyhedra format.

\addtolength{\baselineskip}{-0.5\baselineskip}
\begin{thebibliography}{99}
\bibitem[AF92]{AF} Avis, D.  and Fukuda, K., ^^ ^^ A pivoting algorithm
        for convex hulls and vertex enumeration of arrangements
        and polyhedra," {\em Discrete Comput Geometry} 8
        (1992), 295-313.

\bibitem[CGAF93]{CGAF} Ceder, G., Garbulski, G.D.,  Avis, D. and Fukuda, K., 
^^ ^^ Ground states of a ternary lattice model with nearest
and next-nearest neighbor interactions," preprint  (1993),
Physical Review B, to appear.

\bibitem[G90]{G} Grishukhin, V.P., ^^ ^^ All facets of
the cut cone for n=7 are known," European Journal of
Combinatorics 11 (1990), 115-117.

\bibitem[MRTT53]{MRTT} Motzkin, T.S. , Raiffa, H.,  Thompson, G.L. and  Thrall, R.M., ``The double
description method," in  ``Contribution to the Theory of Games, Vol. II"
(H.W. Kuhn and A.W. Tucker, eds.),
Annals of Math. Studies 28,  Princeton University Press, 1953, pp.81-103.

\end{thebibliography}


\end{document}
